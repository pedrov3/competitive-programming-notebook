\documentclass[a4paper]{article}
\usepackage{xcolor}
% Definindo novas cores
\definecolor{verde}{rgb}{0,0.5,0}
% Configurando layout para mostrar codigos C++
\usepackage{listings}
\lstset{
  language=C++,
  basicstyle=\ttfamily\small,
  keywordstyle=\color{black},
  stringstyle=\color{black},
  commentstyle=\color{black},
  extendedchars=true,
  showspaces=false,
  showstringspaces=false,
  numbers=left,
  numberstyle=\tiny,
  breaklines=true,
  backgroundcolor=\color{white},
  breakautoindent=true,
  captionpos=b,
  xleftmargin=0pt,
}
\pagestyle{empty}
\renewcommand{\contentsname}{Sumario}


\begin{document}
\tableofcontents



\section{Template Longo}
\begin{lstlisting}
#include <bits/stdc++.h>

#include <ext/pb_ds/assoc_container.hpp>
#include <ext/pb_ds/detail/standard_policies.hpp>
#include <ext/pb_ds/tree_policy.hpp>

#define int long long
#define all(x) x.begin(), x.end()

using namespace __gnu_pbds;
using namespace std;

template <class T> using ordered_set = tree<T, null_type, less<T>, rb_tree_tag, tree_order_statistics_node_update>;

% template <class T> using ordered_multiset = tree<T, null_type, less_equal<T>, rb_tree_tag, tree_order_statistics_node_update>;

typedef long long ll;
typedef pair<int, int> ii;
typedef tuple<int, int, int> iii;
typedef vector<int> vi;
const ll oo = 1987654321987654321;

template <class It>
void db(It b, It e) {
  for (auto it = b; it != e; it++) cout << *it << ' ';
  cout << endl;
}

int32_t main() {
  cin.tie(0);
  ios_base::sync_with_stdio(0);
}

\end{lstlisting}



\section{PBS}
\begin{lstlisting}
s.insert(1);
s.insert(2);
s.insert(4);
s.insert(8);
s.insert(16);

cout<<*s.find_by_order(1)<<endl; // 2
cout<<*s.find_by_order(2)<<endl; // 4
cout<<*s.find_by_order(4)<<endl; // 16
cout<<(end(s)==s.find_by_order(6))<<endl; // true

cout<<s.order_of_key(-5)<<endl;  // 0
cout<<s.order_of_key(1)<<endl;   // 0
cout<<s.order_of_key(3)<<endl;   // 2
cout<<s.order_of_key(4)<<endl;   // 2
cout<<s.order_of_key(400)<<endl; // 5
\end{lstlisting}


 
\section{SQRT Decomposition on Trees}

\begin{lstlisting}

int B = ceil(sqrt(h_mx)) + gen(1, 100);
for (int i=0; i < h_mx; i+=B){
    int mi = -1;
    for (int j=0; j < B; j++){
        int id = i + j;
        if (!by_height[id].empty()){
            if (mi == -1 || (by_height[mi].size() > by_height[id].size())){
                mi = id;
            }
        }
    }
    if (mi != -1){
        for (auto &u: by_height[mi]){
            for (auto &v: by_height[mi]){
                if (u <= v){
                    tab[ii(u, v)] = solve(u, v);
                }
            }
        }
        for (auto &u: by_height[mi]){
            special[u] = true;
        }
    }
}
 \end{lstlisting}

\section{Digit DP - Generalizada}
\begin{lstlisting}
#include <bits/stdc++.h>
#define BIT(x, i) ((x & (1ll << i)) ? 1 : 0)
#define int long long

using namespace std;

typedef long long ll;
const ll P = 1e9 + 7;

ll memo[50][2][2][2][2];

ll dp(int i, int upper0, bool e0, int upper1, bool e1, int upper2, bool e2,
      int x = 0) {
  if (i == -1) return x != 0;
  ll &ans = memo[i][e0][e1][e2][x];
  if (ans != -1) return ans;
  ans = 0;
  int r0 = e0 ? BIT(upper0, i) : 1;
  int r1 = e1 ? BIT(upper1, i) : 1;
  int r2 = e2 ? BIT(upper2, i) : 1;
  for (int v0 = 0; v0 <= r0; v0++) {
    bool new_e0 = e0 && v0 == r0;
    for (int v1 = 0; v1 <= r1; v1++) {
      bool new_e1 = e1 && v1 == r1;
      for (int v2 = 0; v2 <= r2; v2++) {
        bool new_e2 = e2 && v2 == r2;
        ans += dp(i - 1, upper0, new_e0, upper1, new_e1, upper2, new_e2,
                  ((x) || (v0 ^ v1 ^ v2)));
        ans %= P;
      }
    }
  }
  return ans;
}

ll solve(int x, int y, int z) {
  memset(memo, -1, sizeof(memo));
  return dp(40, x, true, y, true, z, true);
}

ll expbin(ll a, ll b) {
  if (b == 0LL) return 1LL;
  if (b & (1LL)) return (a % P * expbin(a, b - 1)) % P;
  ll tmp = expbin(a, b / 2);
  return (tmp * tmp) % P;
}

int32_t main() {
  cin.tie(0);
  ios_base::sync_with_stdio(0);

  int l[3], r[3];
  ll q = 1;
  for (int i = 0; i < 3; i++) {
    cin >> l[i] >> r[i];
    q *= (r[i] - l[i] + 1) % P;
    q %= P;
  }
  cout << ((((((((((  solve(r[0], r[1], r[2])
                    - solve(l[0] - 1, r[1], r[2])) %P
                    - solve(r[0], l[1] - 1, r[2])) % P
                    - solve(r[0], r[1], l[2] - 1)) % P
                    + solve(l[0] - 1, l[1] - 1, r[2])) % P
                    + solve(l[0] - 1, r[1], l[2] - 1)) % P
                    + solve(r[0], l[1] - 1, l[2] - 1)) % P
                    - solve(l[0] - 1, l[1] - 1, l[2] - 1)) % P + P) % P)  * expbin(q, P - 2)) % P << "\n";
}
 \end{lstlisting}

\section{BIT 2D}
\begin{lstlisting}
ll bit[MAX][MAX];

void add(int i, int j, ll v) {
  for (; i < MAX; i += i & (-i))
    for (int jj = j; jj < MAX; jj += jj & (-jj)) bit[i][jj] += v;
}

ll getbit(int i, int j) {
  ll sum = 0;
  for (; i; i -= i & (-i))
    for (int jj = j; jj; jj -= jj & (-jj)) sum += bit[i][jj];
  return sum;
}

ll getbit(int lx, int ly, int rx, int ry) {  // getbit(1, 1, lin, col)
  return getbit(rx, ry) - getbit(rx, ly - 1) - getbit(lx - 1, ry) +
         getbit(lx - 1, ly - 1);
}

void add(int lx, int ly, int rx, int ry, ll v) {  // canto superior esquerdo
  rx++;                                           // canto inferior direito
  ry++;
  add(lx, ly, +v);
  add(lx, ry, -v);
  add(rx, ly, -v);
  add(rx, ry, +v);
}
\end{lstlisting}
\section{BIT 2D - Esparsa}
\begin{lstlisting}
ordered_set<ii> bit[MAX];

void insert(int x, int y) {
  for (int i = x; i < MAX; i += i & -i)
      bit[i].insert(ii(y, x));
}

void remove(int x, int y) {
  for (int i = x; i < MAX; i += i & -i)
      bit[i].erase(ii(y, x));
}

int get(int x, int y) {
  int ans = 0;
  for (int i = x; i > 0; i -= i & -i)
      ans += bit[i].order_of_key(ii(y + 1, 0));
  return ans;
}

int get(int lx, int ly, int rx, int ry) {
  return get(rx, ry) - get(rx, ly - 1) - get(lx - 1, ry) + get(lx - 1, ly - 1);
}
 \end{lstlisting}

\end{document}
